\documentclass{article}

\usepackage[utf8]{inputenc}
\usepackage[T1]{fontenc}
\usepackage[slovene]{babel}
\usepackage{hyperref}

\begin{document}

\begin{center}
{\bf \Huge TAROK} \\[24pt]
{\Large Programski projekt pri predmetu programiranje} \\[12pt]
{\Large Jure Slak, vpisna številka: 27121100} \\[12pt]
\today
\end{center}

\tableofcontents
%\newpage

\section{Uvod}
Cilj projekta je implementirati igro s kartami tarok za 4 igralce. Uporabnik bi igral s tremi igralci, ki jih igra računalnik.

\section{Implementacija}
\subsection{Pravila}
Pri taroku obstaja veliko pravil in posebnih primerov. Za lažjo implementacijo sem si jih malo poenostavil, podpiramo samo običajne igre od 1 do 3 s solo variantami, tudi klicanje iger je poenostavljeno. Vsak igralec izbere igro, ki jo zeli, tisti z najvišjo igro zmaga. Točkovanje je zelo podobno, le da ne zaokrožujenmo na 5 točk natančno.

\subsection{Zgradba programa}
Glavna ideja implementacije je, da imamo neko osrednjo strukturo, ki deluje kot strežnik oziroma miza, na katero so povezani 4 igralci. Trije igralci so umetni, s popolnoma enako inteligenco, ćetrti pa je uporabnik. Vse skupaj je zapakirano v en objekt, za lažjo uporabo.

Razred \texttt{GUI} je razred, v katerem so shranjeni strežnik in vsi 4 igralci. Ta razred skrbi za izris na zaslon in kontrolira širši potek igre. Znotraj imamo objekt razreda \texttt{Game}, ki vodi igro in kliče primerne funkcije pri igralcih. Ti objekti morajo, kot je v Pythonu pogosto, slediti principut \emph{duck typing}-a, implementirati morajo naslednje metode:
\begin{itemize}
  \item \texttt{zacniIgro(karte)},
  \item \texttt{zalozi(karte, talon)},
  \item \texttt{zacniRedniDel(idxIgralca, glavniigralec, tipIgre, ostanekTalona, pobraneKarteTalona)},
  \item \texttt{vrziKarto(tvojeKarte, karteNaMizi, prviIgralec)},
  \item \texttt{konecKroga(zmagal, prviIgralecVKrogu, zmagovalec, karteNaMizi)},
  \item \texttt{konecIgre(razlog)}.
\end{itemize}
Pri igralcih, ki jih igra računalnik (razreda \texttt{Igralec}) te metode vrnejo primerne podatke, pri igralcu, ki omogoča igranje uporabniku pa nariše primerne grafične elemente, da lahko uporabnik vnese svojo odločitev (razred \texttt{UserIgralec}).

Vsi razredi imajo veliko skupnih elementov, na primer definicije kart in tipov igre. Take stvari sem zapakiral v knjižnjico \texttt{common}, ki  si jo delijo vsi razredi.


\section{Delovanje}
Program najprej premeša vrstni red igralcev, nato razdeli karte in da igralcem možnost izbire igre. Nato se igralec z najvišjo vrednostjo igre založi, 
pravila za zalaganje so enaka kot pri navadnem taroku. Nato se uporabniku pokažejo informacije o zalaganju in odigra se 12 krogov igre. Po vsakem krogu uporabnik izve, kdo je pobral trenuten krog. Na koncu računalnik sešteje točke kart in prišteje vrednost igre ter te točke napiše igralcu ali igralcema, ki so v tej igri igrali. Uporabniku ponudi novo igro, pri čemer rezultate dosedanjih iger sešteva.

\section{Sklep}
Prostora za izboljšavo je še veliko, lahko bi se bolj dosledno držali pravil taroka ali pa implementirali še kakšno ``posebno'' igro, kot na primer valata ali berača.
To ne bi bilo zelo težko integrirati v program, vendar bi bilo veliko dela, tako da tega nisem implementiral. Prav tako igre ne bi bilo težko igre razširiti na več igralcev.

Celotna koda projekta je objavljena in prosto dostopna na \url{http://github.com/jureslak/tarok/}.


\end{document}
